%!TEX root = handout.tex

\section{Session 1 - Sequence databases}
\label{Session 1 - Sequence databases}
\begin{task}

The following tasks are a good starting point to get used with sequence database handling in Protoemics. Some questions don't have yes / no answers, but need differentiation.
\begin{enumerate}
\item How long are the sequences of human GAPDH (isoform-1) in GenBank and UniProtKB/SwissProt? What is/are the reason(s) for the difference in length?
\item How many sequence isoforms has the human protein "sorcin" in \mbox{UniProtKB}/ SwissProt? In which sequence positions / ranges do they differ? What reason is given for the different isoforms?
\item Are ApoB variants ApoB-48 and ApoB-100 (due to RNA-Editing) stored as sequence isoforms in UniProtKB/SwissProt?
\item Are all SNPs of Glucokinase shown in GenBank (or their amino acid changes in UniProtKB/SwissProt?
\item What is the problem with the regular expression \verb+>sp|\([^|]*\)+ when used for parsing of accessions in uniprot\_complete-mus\_musculus.fasta (here \verb+\(+ and \verb+\)+ are used for accessing the contents of a sub-expression)?
\item How many sequences are contained in the 2015\_9 release of UniProtKB human complete reference  proteome set, both without isoforms and with isoforms?
\item Download the UniprotKB complete reference proteome set of the sea snail Lottia gigantea as FASTA. How many sequences are in the FASTA? How many are from SwissProt and how many from TrEMBL?
\item Why are in the UniProtKB human complete reference proteome set (which contains protein sequences of SwissProt and TrEMBL for all known human genes) less SwissProt sequences than in the human UniProtKB/SwissProt FASTA itself?

\end{enumerate}
\end{task}
