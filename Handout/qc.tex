%!TEX root = handout.tex

\newpage
\section{Quality Control}
\label{Quality Control}

\subsection{Introduction}

\textbf{Q}uality \textbf{C}ontrol is an important part of  mass spectrometry experiments and analyses. However, workflows and quality control protocols may differ vastly between labs. For this reason, quality control in OpenMS is designed to be very flexible, using a userdefined set of metrics from a controlled vocabulary (CV), the QC-CV. By using an own CV, Quality Reports are not limited in its form. Thus, results can be customized and still fit a standard quality report template. Reports are  easy to evaluate due to the design of the used format, qcML.


\subsection{QC-CV}
A controlled vocabulary is a list of entries defining each a phrase or keyword in a given context. In our case, a controlled vocabulary entry of a metric can describe both experimental as well as programmatic environmental variables. It is comprised of a name, an identifier, and a definition. This definition describes the metric and what aspect of quality control is conducted with such a metric. \\
A CV entry also has associated relations, like \textit{Chromatogram count} is a \textit{MS aquisition result details}. This gives the CV a hierarchical structure and should make it easier to comprehend and also easier to browse (\url{http://www.ebi.ac.uk/ontology-lookup/browse.do?ontName=qcML}). \\
Like this, a set of quality metrics can be chosen according to the requirement of the lab, researcher or workflow. The content of the vocabulary is community defined. If your favourite metric, with which you control the quality of your experiments, does not yet have an entry, feel urged to propose its addition. \\
Inside a qcML file, you will have a bunch of quality parameters. These belong to a specific MS run (or set) and combine a value with a CV entry., e.g. $2069$ with \textit{MS2 spectra count}.

\subsection{Single run QC}
\label{Single run QC}

We will start by adapting the label-free quantification workflow as it comprises all the basic analyses we need to begin with:

\begin{itemize}
\item The spectra data itself,
\item the identifications to the spectra and
\item the features found on which quantification is based.
\end{itemize}

Calculating basic statistics and agglomerating quality data will be done by the \KNIMENODE{QCCalculator} node from \menu{Community Nodes > OpenMS > Utilities}. It will also be needed for  more advanced quality metrics later on. The QCCalculator node consumes MS runs (mzML), identifications (idXML), and found features (featureXML). The output are quality parameters which are stored in a qcML file. This will make it easy to access the quality data we need.\\
The qcML file will be handed down from one metric calculation or visualization to another. Each QC step will append its plot, table, or value. With the qcML file you can also instantly visualize the metrics you have calculated so far: qcML includes all information, renderable in all recent browsers, even without an internet connection. \\
At this point, the qcML will not contain much to look at, so we will start filling it - with a heatmap-like (RT over M/Z, Intensity color coded) plot of the mass spectrometry experiment itself. We can extract that from the mzML with an ImageCreator node from \menu{Community Nodes > OpenMS > Utilities}. The plot can then be integrated to the qcML with the QCEmbedder node from \menu{Community Nodes > OpenMS > Utilities}. This node is capable to embed either images or tables into the qcML file as attachment to the quality parameter of a certain run or set. Thus, it needs as parameters the CV accession  of the quality parameter to which it should to be attached to (\textit{qp\_att\_acc}), and a CV accession for the content itself (\textit{cv\_acc}) if there is one available. The image from the ImageCreator we will attach to \textit{QC:0000004}, which is the \textit{MS aquisition result details} accession. If a QualityParameter with that accession does not exist, it will be created for that purpose. And we give itself the CV \textit{QC:0000055} which is the \textit{MS experiment heatmap} accession. We will not specify a run, since in a single run QC we are sure there is just one run to consider and the QCEmbedder will figure that out.\\

\subsubsection{ID ratio}
Next, we will have a look at the ID ratio. Therefore, import the qc\_id\_ratio workflow from the workflow folder and copy that meta node into your workflow to take input from the QCEmbedder. The ID Ratio meta node will create a plot of the measured spectra vs. the identified spectra in a M/Z vs. RT map. If you open the meta node, you will see that we can extract plots or tables just similar to how we could embed them. But this time we handle a table, which will be extracted in csv format. With dedicated KNIME reader nodes, we read the tables into KNIME and calculate the metric with R. It will map the recorded MS2 spectra against the Identifications via the RT and M/Z coordinates of their precursors.\\
The resulting plot will be included in the qcML file and we can move to the next metric.

\subsubsection{Total Ion Current}
Now, we will embed a plot of the total ion current (TIC). As before, import the qc\_tic\_plot workflow and copy the metanode into your workflow to recieve the input qcML from the ID ration meta node. The operation is pretty straight-forward and will yield the total ion current measured on MS1 level.

\subsubsection{Mass Accuracy}
To take a closer look at our identifications mass accuracy, we import the qc\_mass\_accuracy workflow and copy the meta node to recieve input from the last meta node. This time, two plots are being generated. A histogram will show you the distribution of your identification errors. It should resemble a narrow gaussian distribution. \\ 
The other plot will show you the identification errors over time (RT). \\
Both plots will have the smoothed function of the error plotted in red.

\subsubsection{Fractional Mass}
Our last plot will be a look on what features we found. Import and copy the metanode from the qc\_fractional\_mass workflow. Also, copy the extra input node. This input will be an external reference file containing all the potential theoretical masses we could see with the given experiment (\directory{QC / theoretical\_masses.txt}). It will plot these on nominal vs. fractional mass in blue and the found feature centroids in red. So a real peptide feature should be located inside these blue clouds.

\subsubsection{Final preparations}
After we have all the plots embedded and calculated all the QC values we want, we can get rid of the verbose or residual data we were collecting in order to be able to calculate everything we want. The QCShrinker node from \menu{Community Nodes > OpenMS > Utilities} will take care of that. As input, use the mzML files from \directory{OpenMS / small}. 


\subsection{Inspect the quality reports}
\label{Inspect the quality reports}

As we have a ZipLoop in our labelfree quantification workflow, creating a QC report of several runs is easy. If you connect the last QC node with a new input port of the ZipLoopEnd node, you can take the respective ouput port and connect it with a new Output Folder, select a destination and have a look at the QC reports by opening the qcML files in a browser. 

The final workflow, as it should be build in this section, can be seen in \cref{fig:qc_wf}.

%...with a QCMerger node from \menu{Community Nodes > OpenMS > Utilities} which will create a QC set from the list of qc files (with each a single MS run QC), you will get from the ZipLoopEnds qcML output port. You can configure its parameter setname to reflect the comprised runs and/or give it an individual name.

\begin{figure}[htbp]
  \centering
  \includegraphics[width=\textwidth]{graphics/qc_wf}
  \caption{Complete quality control workflow.}
  \label{fig:qc_wf}
\end{figure}

\todo{clear ambiguous edges in workflow image}


